\begin{landscape}
	\section{Operacionalización de variables}
	\begin{longtable}{p{3cm}p{5.2cm}p{4.4cm}p{3.1cm}p{4.8cm}p{1.5cm}}
		\caption{Matriz de operacionalización} \\
		\toprule
		\textbf{Variables} & \textbf{Definición conceptual} & \textbf{Definición operacional} & \textbf{Dimensiones} & \textbf{Indicadores} & \textbf{Items} \\
		\midrule
		\endfirsthead
		
		\multicolumn{6}{c}%
		{{\tablename\ \thetable{} -- Continuación}} \\
		\toprule
		\textbf{Variables} & \textbf{Definición conceptual} & \textbf{Definición operacional} & \textbf{Dimensiones} & \textbf{Indicadores} & \textbf{Items} \\
		\midrule
		\endhead
		
		\midrule
		\multicolumn{6}{r}{{Continúa en la siguiente página}} \\
		\endfoot
		
		\bottomrule
		\endlastfoot
		
		Medidas de desempeño & Implica analizar el desempeño de las actividades del proyecto en relación con el plan inicial, evaluando el progreso del proyecto en cuanto al logro de sus objetivos, cumplimiento de plazos y presupuesto. & La técnica de Gestión del Valor Ganado implica la medición del desempeño del proyecto recolectando información secundaria de las revisiones documentales de las liquidaciones de obras realizadas en el periodo 2018-2022 del Gobierno Regional de Apurímac  & Medidas del cronograma & Variación del Cronograma (Schedule Variance, SV) & 1 \\
		&       &       &       & Índice de Rendimiento del Cronograma (Schedule Performance Index, SPI) & 2 \\
		&       &       & Medidas del costo & Variación del Costo (Cost Variance, CV) & 3 \\
		&       &       &       & Índice de Rendimiento del Costo (Cost Performance Index, CPI) & 4 \\
		&       &       & Pronósticos & Estimado hasta concluir (Estimate to Complete, ETC) & 5 \\
		&       &       &       & Estimado a la Conclusión (Estimate at Completion, EAC) & 6 \\
		&       &       &       & Variación a la conclusión (Cost variance at completion,VAC) & 7 \\
		&       &       &       & Índice de desempeño del trabajo por completar ( To Complete Performance Index,TCPI) & 8 \\
		Modalidades de ejecución & Las modalidades de ejecución se rigen por normativas específicas que establecen los marcos legales y procedimientos a seguir. Dos de las modalidades más relevantes son la ejecución por administración directa y la ejecución por contrata, las cuales se encuentran reguladas por la Ley de Contrataciones del Estado (Ley N° 30225) y su respectivo Reglamento. & No se recopilarán datos ya que se tratan de variables categóricas & Administración indirecta & Ley de contrataciones del Estado N° 30225 &  \\
		&       &       & Adminitración directa & Resolución de contraloría N° 195-88-CG &  \\
	\end{longtable}
\end{landscape}