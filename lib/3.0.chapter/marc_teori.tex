\chapter{MARCO TEÓRICO REFERENCIAL}
\section{Antecedentes}
% Necesitas saber cuánto se ha estudiado las variables
% Antecedentes mundiales
\cite{JimenezRios2023} realizó un estudio titulado: Learning from the Past: Parametric Analysis of Cob Walls en la ciudad Oslo del país Irlanda.El objetivo general del estudio fue explorar las ventajas y desventajas de cada uno de ellos y proporcionar alternativas a los profesionales: análisis FEM elástico lineal y análisis límite cinemático con macro-elementos. La muestra estuvo constituida por muros de adobe o "cob" como ellos de denominan en Irlanda. El diseño que se utilizó fue experimental. Los instrumentos que se usaron fueron el modelado FEM,  siguiendo dos enfoques computacionales, a saber, el método de elementos finitos y el análisis límite cinemático.Y los resultados obtenidos demostraron que los muros tradicionales de barro en Irlanda son muy robustos. Se requerirían valores de aceleración relativamente altos, poco probables en una región con bajo riesgo sísmico como Irlanda.

\cite{Wang2023} realizó un estudio titulado: Adobe-Brick-Masonry Composite Wall with a Wooden-Construction Center Column en la ciudad Shihezi del país China.El objetivo general del estudio fue proponer una tecnología para mejorar el desempeño sísmico de un muro compuesto de albañilería de adobe modificado con una columna central de construcción en madera La muestra estuvo constituida por muro de adobe modificado.El diseño que se utilizó fue experimental.Los instrumentos que se usaron fueron  pruebas de carga cíclica quasiestáticas en plano para estudiar sus modos de fallo e indicadores de desempeño sísmico en un software y los resultados obtenidos han sido mejoraron su capacidad de carga sísmica gracias al barro modificado por otra parte la columna central y los materiales de amarre proporcionaron una segunda línea de defensa que aumentó la ductilidad de los muros y la zona residual de colapso.

\cite{Romanazzi2023} realizó un estudio titulado:  Effectiveness of a TRM solution for rammed earth under in-plane cyclic loads en la ciudad Braga y Guimarães del país Portugal.El objetivo general del estudio fue evaluar la efectividad de una solución de refuerzo TRM para paredes de tierra apisonada sometidas a cargas cíclicas en el plano.una pared de tierra apisonada de un edificio tradicional de un solo piso con techo de madera y geometría en forma de I en planta. El diseño que se utilizó fue experimental.Los instrumentos que se usó fué protocolo de prueba del modelo reforzado GeoRE-IP Y los resultados obtenidos destacaron la efectividad de la solución TRM en la mejora de la capacidad de corte en el plano, la ductilidad y la energía disipada del modelo.

% Antecedentes latinoamericanos
\cite{LopezP.2020} realizó un estudio titulado: Comportamiento sísmico de edificaciones de tapia pisada reforzadas con marcos de madera y viga de coronación en concreto, en la ciudad Bogotá del país Colombia.El objetivo general del estudio fue reducir esta vulnerabilidad.La muestra estuvo constituida por  ensayos en mesa vibratoria en un modelo a escala 1:6 y su contraparte no reforzada .El diseño que se utilizó fue experimental.Los instrumentos que se usaron fueron Implementación de un diafragma rígido y los resultados obtenidos han sido  muestran una mejora significativa en el comportamiento global de la construcción (reducción en los desplazamientos hasta del 80\% en los modelos reforzados), logrando el objetivo de prevenir el colapso, manteniendo la integridad de la estructura.

\cite{TorresMoreno2022} realizó un estudio titulado: Influencia de la resistencia a la compresión y módulo de elasticidad del adobe en el comportamiento estructural de viviendas patrimoniales en la ciudad Riobamba del país Ecuador.El objetivo general del estudio fue obtener una base de datos de la resistencia a la compresión y módulo de elasticidad del adobe, y determinar su influencia en el comportamiento estructural de tres viviendas patrimoniales en el cantón Guamote. La muestra estuvo constituida por 12 muestras de adobe de 3 viviendas diferentes El diseño que se utilizó fue experimental-correlacional. Los instrumentos que se usaron fueron ensayos en el laboratorio  los resultados obtenidos muestran que las resistencias a la compresión que van de 0.062 a 0.1852 MPa y módulos de elasticidad de 0.9012 a 2.1195MPa. 

\cite{AmaguayBermeo2022} realizó un estudio titulado: Evaluación y reforzamiento estructural, incorporando mampostería enchapada y alternativa para el mejoramiento de suelo o refuerzo de cimentación de una edificación que presenta asentamientos diferenciales en la ciudad de Quito del país Ecuador.El objetivo general del estudio fue evaluación estructural a una edificación de departamentos ubicada en la parroquia de la Magdalena.La muestra estuvo constituida por una sola vivienda de mampostería.El diseño que se utilizó fue no experimental Los instrumentos que se usaron fueron ficha de recolección de datos, mediante una inspección visual y los resultados obtenidos han sido que la vivienda en estudio presenta una alta vulnerabilidad e implementó un reforzamiento enchapado.

% Antecedentes nacionales y regionales
\cite{OteroMonteza2022} realizó un estudio titulado: Influence of the aspect ratio on seismic performance of adobe buildings en la ciudad de Lima del país Perú. El objetivo general del estudio fue el comportamiento sísmico de construcciones de adobe con muros más grandes en un eje que en el otro eje ortogonal. La muestra estuvo constituida por cuatro modelos de edificios con diferentes relaciones de aspecto, que varían en longitud desde los ocho metros hasta los cincuenta y dos metros.  El diseño que se utilizó fue experimental Los instrumentos que se usaron fueron análisis de historia del tiempo no lineal con tres registros sísmicos peruanos Y los resultados obtenidos han sido Muestra los modos fundamentales de vibración en la dirección del eje y el primer modo de vibración involucra el 48.48\% de la masa total.

\cite{QuirozHuaraya2021} realizó un estudio titulado: Propuesta de reforzamiento estructural de viviendas de adobe utilizando mallas de acero electrosoldadas en Huarangal, Carabayllo en la ciudad de Lima del país Perú.El objetivo general del estudio fue proponer un tipo de reforzamiento para muros de adobe con el uso de mallas electrosoldadas.La muestra estuvo constituida En la primera fase se realizaron ensayos de compresión axial en tres pilas de adobe y compresión diagonal en tres muretes de adobe sin reforzamiento. En la segunda fase se ensayaron 12 muretes de adobe reforzados con dos tipos de mallas electrosoldadas y en diferentes lados de los especímenes. El diseño que se utilizó fue experimental.Los instrumentos que se usaron fueron ensayo de mueretes con mallas electrosoldadas y sin mallas y los resultados obtenidos han sido los ensayos demostraron un incremento en la resistencia al corte y una mejora en la ductilidad de los muretes de adobe reforzados debido al tipo de malla y al número de lados reforzados en cada murete.

\cite{CerveraTimana2023} realizó un estudio titulado: Evaluación de la vulnerabilidad sísmica en las edificaciones de la zona sur-este del distrito de Lambayeque en la ciudad de Chiclayo del país Perú.El objetivo general del estudio fue determinar los niveles de vulnerabilidad sísmica en las edificaciones existentes de la zona sur-este del distrito de Lambayeque. La muestra estuvo constituida por se evaluaron 3,054 edificaciones.El diseño que se utilizó fue no experimental Los instrumentos que se usaron fueron Método del Índice de Vulnerabilidad de Benedetti - Petrini y los resultados obtenidos muestran que 477 edificaciones resultaron con baja vulnerabilidad representando un 15.62\%; 1,901 edificaciones resultaron con vulnerabilidad media representando un 62.25\%; y 676 edificaciones resultaron con alta vulnerabilidad representando un 22.13\%.

\section{Marco teórico}

\cite{CENEPRED2014}

\section{Marco conceptual}