\chapter{METODOLOGÍA}
\section{Tipo y nivel de investigación}
\subsection{Tipo de la investigación}

Se presentan diversas perspectivas con respecto a la clasificación o división de la investigación. No obstante, independientemente del criterio utilizado, según \cite[]{Zacarias2020} se espera que cumpla con el principio de parsimonia, es decir, que sea exhaustivo y excluyente.

\cite[47]{SanchezCarlessi2015} destaca la importancia de estar conscientes sobre la naturaleza y los propósitos de la investigación, clasificándolos en tres categorías. En primer lugar, se encuentra la investigación básica, también conocida como pura o fundamental, según lo menciona el autor. En este tipo de investigación, el investigador sostiene que \emph{"[...] Mantiene como propósito recoger información de la realidad para enriquecer el conocimiento científico, está orientado al descubrimiento de principios y leyes”}. En segunda instancia, nos encontramos con la investigación aplicada, también denominada constructiva o utilitaria. En este caso, la investigación aplica los conocimientos teóricos a una situación específica para obtener consecuencias prácticas derivadas de la deducción. Por último, se aborda la investigación sustantiva, donde el enfoque está dirigido a describir, explicar y prever la realidad.

Por otra parte, \cite[36]{Tamayo2004} clasifica la investigación según su enfoque. Por un lado, se encuentra la investigación pura, también denominada básica o fundamental. Esta modalidad se caracteriza por buscar confrontar la teoría con la realidad, con la finalidad de desarrollar una teoría basada en leyes y principios. Por otro lado, está la investigación aplicada, también conocida como activa o dinámica. Su particularidad radica en que se centra en la aplicación inmediata de conocimientos, sin enfocarse en el desarrollo de teorías.

Ahora bien, al considerar la revisión bibliográfica de los dos autores, se determina que la presente investigación adopta un enfoque básico, puro o fundamental. En este contexto, se llevará a cabo la recopilación de datos existentes en el Gobierno Regional de Apurímac.

\subsection{Nivel de la investigación} \label{nivel_invest}

\cite[88]{RiosRamirez2017} clasifica el nivel de investigación según los propósitos o finalidades, la naturaleza de los datos o enfoque, el nivel de conocimiento, las fuentes de información, las condiciones de información, el tiempo y el diseño o control de las variables. En este caso, es necesario profundizar en el criterio del nivel de conocimiento, según el autor, quien afirma que "[...] involucra el grado de conocimiento sobre el objeto de estudio". Al mismo identifica cuatro tipos de investigación según esta clasificación. El primero es el exploratorio, un estudio en el cual no se puede distinguir claramente entre las variables independientes o dependientes, de ahi que también se le conoce como univariado. En este tipo de investigación, se examinan temas poco abordados. El descriptivo, por su parte, "[...] busca encontrar las características, comportamiento y propiedades del objeto de estudio" en el tiempo presente o futuro. La investigación relacional hace mérito a su nombre, mide la relación entre dos o más variables dadas, pero es importante señalar que no determina la causa-efecto. Por último, la investigación explicativa sí determina la relación causa-efecto.

Por otra parte, si bien es cierto que \cite[93]{HernandezSampieri2014} coincide con \citeauthor{RiosRamirez2017} en la mayoría de las denominaciones de su clasificación, el autor utiliza la denominación "correlacional" en lugar de "relacional". Muchos autores como \cite[]{Zacarias2020} opinan que debería llamarse "relacional", ya que la relación implica una conexión entre variables, mientras que la correlación se refiere a la relación entre unidades.

\section{Diseño de la investigación}
\cite[128]{HernandezSampieri2014} explica el diseño de la investigación como el "plan o estrategia que se desarrolla para obtener la información requerida en una investigación". El autor distingue dos diseños, ambos con una importancia destacada. 

En primer lugar, está el diseño experimental, donde las variables independientes se manipulan deliberadamente para observar sus efectos sobre otras variables dependientes en una situación controlada. A su vez, este diseño se subdivide en tres clases. La característica distintiva de todos ellos es que deben tener validez interna y externa, es decir, que los resultados deben ser aplicables primero a la población en estudio y luego al entorno externo. Las tres subclases son: preexperimentos, experimentos puros y cuasiexperimentos. No profundizaremos en estos enfoques ya que no es el objetivo de este estudio.

En contraste, los diseños no experimentales son aquellos estudios en los cuales solo se observan los fenómenos, y no se realiza una manipulación deliberada de variables. Este tipo de diseño, según su dimensión temporal o puntos en los cuales se recolectan datos, se puede clasificar en investigación transeccional, también conocida como transversal, la cual se caracteriza por recopilar datos una sola vez y en un solo momento. A su vez, se clasifica en tres subcategorías: exploratorios, descriptivos y correlacionales-causales, basándose en lo mencionado en la sección \ref{nivel_invest}. Por otro lado, la investigación longitudinal o evolutiva obtiene datos en diferentes puntos en el tiempo, ya que compara los resultados a través del cambio.\cite[341]{NaupasPaitan2014} clasifica este diseño en descriptiva simple, descriptiva comparativa, causal-comparativa, correlacional, longitudinal y transversal.

En cuanto a los criterios de planificación para la toma de datos en el tiempo, no se encontraron autores que mencionen explícitamente esta clasificación. Sin embargo, se sobreentiende que si los datos fueron planificados previamente, el estudio será prospectivo. En cambio, si los datos ya estaban registrados, el estudio será retrospectivo.

\section{Descripción ética de la investigación}

Según \cite[]{AbreuSuarez2017}, la ética se presenta como un campo que abarca los valores esenciales del ser humano, como la honestidad, la solidaridad, el respeto y la tolerancia, entre otros aspectos. Estos elementos deben ser considerados por los investigadores.

El investigador aseguró obtener el consentimiento informado de todas las partes involucradas en la investigación, abarcando tanto a los funcionarios del Gobierno Regional de Apurímac como a cualquier otra persona que pudiera ser afectada directa o indirectamente por los resultados del estudio.

Se garantizó la confidencialidad de los datos y la privacidad de los participantes en el estudio. La información recopilada fue manejada de manera segura y exclusivamente utilizada con fines investigativos específicos, preservando la identidad y la información personal de los involucrados.

Manteniendo la imparcialidad y objetividad en todas las etapas del estudio, el investigador basó los resultados en el análisis de datos precisos y fiables, evitando sesgos y asegurando una interpretación justa y equitativa de los hallazgos.

La investigación se llevó a cabo con honestidad y rigor académico, evitando la fabricación, falsificación o manipulación de datos, así como cualquier forma de plagio de trabajos previos.

Considerando el equilibrio entre beneficios potenciales y riesgos asociados con el estudio, se garantizó que los posibles beneficios superaran cualquier daño potencial a los participantes o a la comunidad en general.

Dada la naturaleza que involucra seres humanos o datos sensibles, el investigador obtuvo la aprobación de la Subgerencia de Obras del Gobierno Regional de Apurímac, entidad encargada de supervisar la integridad y el bienestar de los participantes.

El investigador practicó la transparencia en la divulgación de métodos, resultados y conclusiones, incluyendo la revelación de cualquier conflicto de intereses y proporcionando los datos subyacentes para permitir la validación y replicación del estudio por parte de otros.

Los resultados de la investigación serán utilizados de manera responsable y ética, considerando sus implicaciones y su posible impacto en la toma de decisiones y políticas públicas.

Se mostró respeto y valoración hacia la cultura, tradiciones y conocimientos locales, evitando cualquier forma de apropiación indebida de la información recopilada.

En la medida de lo posible, la investigación buscó contribuir al bienestar social y al desarrollo de la comunidad, proporcionando información relevante y útil para mejorar la gestión de proyectos de inversión pública en el Gobierno Regional de Apurímac.

El presente trabajo de investigación en cuanto a las citaciones de referencias bibliográficas sigue las directrices de las normas establecidas por \cite[]{IOS2021}, mas conocidos como ISO-690 , en cuanto a la redacción siguen las pautas proporcionadas por el formato del Vicerrectorado de investigación de la Universidad Nacional Micaela Bastidas de Apurímac.

Por otra parte se tomó en cuenta con lo dispuesto por el \cite[]{VRI2018}, los titulos que corresponden a las normas de comportamiento de quienes investiguen en la página 2, buenas prácticas de los investigadores y la investigación con personas.
