\pagestyle{fancy}
\chapter{PLANTEAMIENTO DEL PROBLEMA}
\section{Descripción del problema}
\lipsum[4-5]
\lipsum[4-5]
\lipsum[4-8]
\section{Enunciado del problema}
\subsection{Problema general}
\subsection{Problemas específicos}
\section{Justificación de la investigación}
\begin{minipage}{\textwidth}
	
	\chapter{DATOS GENERALES DEL ALUMNO}
	\noindent \textbf{Apellidos y nombres:} Quintana Arone Juan Benito \\
	\textbf{Código}: 152254\\
	\textbf{Último ciclo de estudios terminado:} 2023-I\\
	\textbf{Créditos aprobados:} 203 créditos\\
	\textbf{Docente asesor:} Ph.D, Ing. Lucy Marisol Guanuchi Orellana\\
	\chapter{DATOS GENERALES DE LA ENTIDAD PÚBLICA}
   
	\section{Hola}		
	\begin{mdframed}[
		backgroundcolor=white,
		roundcorner=10pt,
		linewidth=0pt,
		innerleftmargin=0.0\textwidth,
		innerrightmargin=0.0\textwidth,
		skipabove=\topsep,
		skipbelow=\topsep,
		rightline=false,
		rightmargin=0.0\textwidth,
		leftmargin=0.05\textwidth,
		splittopskip=\topskip,
		splitbottomskip=\topskip,
		align=left,
		nobreak=false,
		]
        \subsection{Nombre}
		Municipalidad distrital de Tamburco, ubicado en la provincia de Abancay, departamento de Apurímac.
	
		Creado mediante Ley del 31 de diciembre de 1941, en el Primer gobierno de Manuel Prado Ugarteche. Su nombre deriva de las voces quechuas Tambo (descanso) y Orcco (cerro); siendo así semánticamente Cerro de descanso, en efecto en la etapa colonial del Perú, Tamburco era un centro de descanso y abastecimiento	
			
	\end{mdframed}
	
\end{minipage}